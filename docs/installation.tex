\htmlhr
\chapter{Installation and app setup\label{installation}}
This chapter describes how to install the SPARTA tools
(Section~\ref{sec:install}) and how to prepare an Android App to have the
SPARTA tools run on it (Section~\ref{sec:antsetup}).

\section {Requirements\label{sec:requirements}}
\paragraph{Java 7}
\begin{itemize}
 \item  \<.../jdk1.7.0/bin> must be on your path.
 \item \<JAVA\_HOME> should be set to \<.../jdk1.7.0>.
\end{itemize}

\paragraph{Ant}
\begin{itemize}
 \item Ant version 1.8.2 or later
\end{itemize}

\paragraph{Android SDK}
\begin{itemize}
 \item Install the Android SDK to some directory. 
 \item Set \<ANDROID\_HOME> to the directory where you installed the
   Android SDK.
 \item Download the \<android-15> target by running \<\$ANDROID\_HOME/tools/android>
\end{itemize}

If using Eclipse, go to
\<Help $\rightarrow$ Install New Software>
and install the Android ADT Plugin (\url{https://dl-ssl.google.com/android/eclipse}).

\paragraph{Checker Framework}
\begin{itemize}
\item Follow the installation instructions in the manual: 
\url{http://types.cs.washington.edu/checker-framework/current/checkers-manual.html#installation}
\item As described in the installation instructions, set the \<CHECKERS>
  environment variable to \<.../checker-framework/checkers/>
\item Developers on the SPARTA project should follow Section 25.3 from the Checker Framework manual (Building from source.)
\end{itemize}


\section{Install SPARTA\label{sec:install}}

\begin{enumerate}

\item
  Obtain the source code for the SPARTA tools, either from its version
  control repository or from a packaged release.

\begin{itemize}
\item
  To obtain from the version control repository, run
\begin{Verbatim}
 hg clone https://dada.cs.washington.edu/hgweb/sparta-code
\end{Verbatim}
  using the credentials you have been given.
\item 
  If you do not have access to the source code repository, then
  download the SPARTA release from
  \url{http://types.cs.washington.edu/sparta/release/}.  (Please do not
  publicize this URL.)  

  Then, unpack the archive.
\end{itemize}

%%For now, we don't support the JSON feature, but there is still a dependency 
 \item
 Google gson is a dependency for the \<\emph{*}-json> ant targets.
 
 \begin{itemize}
 \item
 If Google gson is already installed, then set build property \<gson.jar>, which defaults to:
 \begin{alltt}
 gson.jar=${basedir}/lib/google-gson-2.2.2/gson-2.2.2.jar
 \end{alltt}
 \item
 If Google gson is not already installed, then 
 get it from \url{http://code.google.com/p/google-gson/};
 create directory \<sparta-code/lib> and unzip gson there.
 \end{itemize}

%sparta-code isn't an android project, so both of these commands fail with
%this error:
%sparta-code is not a valid project (AndroidManifest.xml not found).
%\item Update the Android development environment, by 
%running the following:
%
%\begin{Verbatim}
%ant -buildfile $SPARTA_CODE/build.local.xml
%\end{Verbatim}
%
%Alternatively, you can run:
%
%\begin{Verbatim}
%$ANDROID_HOME/tools/android update project -{}-path . -{}-target android-15
%\end{Verbatim}
%
%
%Rationale:  When working with Android, a developer must ``update a project'' to
%properly set the path to the Android SDK\@.  For more details about
%updating an Android project, see
%\url{http://developer.android.com/tools/projects/projects-cmdline.html#UpdatingAProject}.


\item Build the SPARTA tools by compiling the source code:
\begin{alltt}
ant jar
\end{alltt}

\item
As a sanity check of the installation, run

\begin{alltt}
ant all-tests
\end{alltt}

You should see ``\<BUILD SUCCESSFUL>'' at the end.


%\noindent
%It is necessary to specify the Android SDK location, by setting the \<ANDROID\_HOME>
%environment variable or the \<android.home> property.  Here is an example
%of the latter:
%
%\begin{alltt}
%ant -Dandroid.home=... jar
%\end{alltt}
%
%See file \<build.properties> for other configuration properties.
%
%See the output of \<ant -p> for the build and test targets.
%
%All projects can also be built and tested in Eclipse.
%Import the annotation-tools, jsr308-langtools, checkers, javaparser,
%and sparta-code projects into a workspace.

\end{enumerate}


\section{Android App Setup\label{sec:antsetup}}

This section explains how to set up an Android application for analysis with the SPARTA tools.

\begin{enumerate}
\item
Ensure the following environment variables are set. 

\begin{itemize}
\item
\<CHECKERS> is the
\<.../checker-framework/checkers> directory

\item
\<SPARTA\_CODE> is the \<.../sparta-code> directory

\item
\<ANDROID\_HOME> is the \<.../android-sdk> directory

\end{itemize}

\item
If your Android project does not have a build.xml file, update the project.

\begin{Verbatim}
$ANDROID_HOME/tools/android update project --path .
\end{Verbatim}

\item
Add the SPARTA build targets to the end of the \<build.xml>
 file, just above \verb|</project>|.


\begin{Verbatim}
<property environment="env"/>
<dirname property="checkers_dir" file="\$\{env.CHECKERS\}"/>
<basename property="checkers_base" file="\$\{env.CHECKERS\}"/>
<dirname property="sparta-code_dir" file="\$\{env.SPARTA_CODE\}"/>
<basename property="sparta-code_base" file="\$\{env.SPARTA_CODE\}"/>
<import file="\$\{sparta-code_dir\}/\$\{sparta-code_base\}/build.include.xml" optional="true"/>
\end{Verbatim}

\end{enumerate}


To use Eclipse to look at and build the code, perform these simple
steps:
\begin{itemize}
\item
Using Eclipse, import the projects (this requires the app to have a
\<.project> and \<.classpath> file)
  \begin{itemize}
    \item
    Make sure
    \<Project Properties $\rightarrow$ Android $\rightarrow$ Android
    version \#> is checked

    \item
    Check that
    \<Project Properties $\rightarrow$ Java Build Path $\rightarrow$
    Libraries $\rightarrow$ Android version \#> appears

    \item
    Add the sparta-code project to
    \<Project Properties $\rightarrow$ Java Build Path $\rightarrow$ Projects>
    
  \end{itemize}

\item Compile via command line (\<ant clean>, \<ant flowchecker>)

\item If it compiles, or the errors are exclusively about annotations,
  it's working correctly.
\end{itemize}

Most Android apps will rely on an auto-generated \<R.java> file
in the \</gen> directory of the project. This will only be generated
if there are no errors in the project. There may be errors in the
resources (\<.../res> directory) that could cause \<R.java> to not be
generated.

Additionally, if the app depends on an external \<.jar> file (often
located in the \<lib/> directory), it will compile in Eclipse but not
with Ant. To fix this, in ant.properties, add ``\<jar.libs.dir=lib>''
(or wherever the \<.jar> is located).



%%% Local Variables: 
%%% mode: latex
%%% TeX-master: "manual"
%%% TeX-command-default: "PDF"
%%% End: 

%  LocalWords:  hg SDK ADT Plugin MercurialEclipse gson json sparta jsr308
%  LocalWords:  Dandroid langtools javaparser app TODO xml env dirname dir
%  LocalWords:  basename
