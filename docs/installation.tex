\htmlhr
\chapter{Installation\label{installation}}

This chapter briefly describes how to install the SPARTA project tools.

\section{Compiling the SPARTA code}

\begin{enumerate}

\item Required programs:

\begin{itemize}
\item Download and install Java 7. Add \<.../jdk1.7.0/bin> to the
  beginning of the PATH variable and set \<JAVA\_HOME> to
  \<.../jdk1.7.0>.

\item Ensure you have recent versions of ant and Mercurial (hg) installed;
  ant version 1.8.2 and Mercurial version 2.0.2 are known to work.

\item Install the Android SDK to some directory. Set \<ANDROID\_HOME> to that
  location.
  Download the \<android-16> target.

\item If using Eclipse, go to
\<Help $\rightarrow$ Install New Software>
and install the Android ADT Plugin (\url{https://dl-ssl.google.com/android/eclipse}) and MercurialEclipse (\url{http://cbes.javaforge.com/update}).

\end{itemize}


\item Install the Checker Framework tool set.

Linux is our development platform, but we know of successful uses on
MacOS; Windows is possible, but takes more effort.

Follow the instruction at:

\url{http://types.cs.washington.edu/checker-framework/current/checkers-manual.html#build-source}

Note: As of writing this (08/24/12), the annotation-tools test cases
produce some error output. This is expected and can be ignored.


\item Install and compile the SPARTA type checkers.

Google gson is a dependency for the "-json" targets for projects.
Get it from \url{http://code.google.com/p/google-gson/};
create directory \<sparta-code/lib> and unzip gson there.
Alternatively, set build property \<gson.jar>, which defaults to:

\begin{alltt}
gson.jar=${basedir}/lib/google-gson-2.2.2/gson-2.2.2.jar
\end{alltt}
% make Emacs $ happy

In the \<sparta-code> directory, run

\begin{alltt}
$ ant jar
\end{alltt}
% make Emacs $ happy

to build the \<sparta.jar> file.
The Android SDK location is either taken from the \<ANDROID\_HOME>
environment variable or can be set with the \<android.home> property:

\begin{alltt}
$ ant -Dandroid.home=...
\end{alltt}
% make Emacs $ happy

See file \<build.properties> for other configuration properties.

All projects can also be built and tested in Eclipse.
Import the annotation-tools, jsr308-langtools, checkers, javaparser,
and sparta-code projects into a workspace.


\item Run test cases.

As a sanity check of the installation, run

\begin{alltt}
$ ant all-tests
\end{alltt}
% make Emacs $ happy

You should see ``\<BUILD SUCCESSFUL>'' at the end.
\end{enumerate}


\section{Setting up a new application to test}
% TODO: should this stay here, or go to apanalysis.tex?

Checking applications

It is required to set three environment variables:

\begin{itemize}
\item
\<CHECKERS> pointing to the
\<.../checker-framework/checkers> directory

\item
\<SPARTA\_CODE> pointing to the \<.../sparta-code> directory

\item
\<ANDROID\_HOME> pointing to the \<.../android-sdk> directory

\end{itemize}


Update the project with your Android settings:

\begin{alltt}
$ ant -buildfile $SPARTA_CODE/build.local.xml>
\end{alltt}

Alternatively, you can run:

\begin{alltt}
$ $ANDROID_HOME/tools/android update project --path . --target android-16>
\end{alltt}

Edit the \<build.xml> file of the project under analysis to add the
SPARTA build targets at the end (right before the ``$<$/project$>$''):
% TODO: how does using lt and gt in TT mode work?

\begin{alltt}
<property environment="env"/>
<dirname property="checkers_dir" file="${env.CHECKERS}"/>
<basename property="checkers_base" file="${env.CHECKERS}"/>
<dirname property="sparta-code_dir" file="${env.SPARTA_CODE}"/>
<basename property="sparta-code_base" file="${env.SPARTA_CODE}"/>
<import file="${sparta-code_dir}/${sparta-code_base}/build.include.xml"/>
\end{alltt}



To use Eclipse to look at and build the code, perform these simple
steps:
\begin{itemize}
\item
Using Eclipse, import the projects (this requires the app to have a
.project and .classpath file)
  \begin{itemize}
    \item
    Make sure
    \<Project Properties $\rightarrow$ Android $\rightarrow$ Android
    version \#> is checked

    \item
    Check that
    \<Project Properties $\rightarrow$ Java Build Path $\rightarrow$
    Libraries $\rightarrow$ Android version \#> appears

    \item
    Add the sparta-code project to
    \<Project Properties $\rightarrow$ Java Build Path $\rightarrow$ Projects>
  \end{itemize}

\item Compile via command line (\<ant clean>, \<ant flowtest>)

\item If it compiles, or the errors are exclusively about annotations,
  it's working correctly.
\end{itemize}

Most (all?) Android apps will rely on an auto-generated \<R.java> file
in the \</gen> directory of the project. This will only be generated
if there are no errors in the project. There may be errors in the
resources (\<.../res> directory) that could cause \<R.java> to not be
generated.

Additionally, if the app depends on an external \<.jar> file (often
located in the \<lib/> directory), it will compile in Eclipse but not
with Ant. To fix this, in ant.properties, add ``\<jar.libs.dir=lib>''
(or wherever the \<.jar> is located).

