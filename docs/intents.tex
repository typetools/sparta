\htmlhr

\newcommand{\tp}[1]{\mathit{typeof}}
\newcommand{\Fix}[1]{\textbf{[[}{\color{red} #1}\textbf{]]}}
\newcommand{\theIntentChecker}{the Intent Checker\xspace}
\newcommand{\TheIntentChecker}{The Intent Checker\xspace}
\newcommand{\sendIntent}{\<sendIntent>}
\newcommand{\onReceive}{\<onReceive>}


\chapter{Intent Checker\label{intent-checker}}

This chapter describes \theIntentChecker{},
which type-checks information flows across communicating components of an
Android app.

Android intents are used for
communication within an app, among apps, and with the Android system.
Intents can be seen as messages exchanged between Android components. Sensitive
data can flow in and out of intent objects. Consequently, to detect forbidden
flows derived from inter-component communication, \theIntentChecker{} needs to
identify the information flow types of the data in an intent.

To use \theIntentChecker{}, a programmer must supply two types of information:
\begin{itemize}
\item
Information about inter-component communication across apps for the target
program. Section \ref{component-map} shows how to compute that
information and supply it to the Intent Checker.
\item
Type qualifiers used to express the data types in an intent. Section
\ref{intent-types} describes the type qualifiers used by \theIntentChecker.
\end{itemize}

Sending an intent is similar to an ordinary method call,
where the data in the intent are the method's arguments. 
\TheIntentChecker{} verifies that for any sending intent method
call and its matching receiving intent method declarations,
the intent argument of the caller is compatible with that of the corresponding
callee. 

An Android component can send a message to an Activity by calling the method \<startActivity>, a
Service by calling the method \<startService>, or a BroadcastReceiver by
calling \<sendBroadcast>. Intents are received in an Activity by the method
\<setIntent>, in a Service by the method \<onBind> or
\<onStartCommand>, and in a BroadcastReceiver by the method
\<onReceive>. For simplicity in this manual, we abstract all methods
that send an intent as the method call \sendIntent{}, and all methods that
receive an intent as the method declaration \onReceive{}.

\section{Inter-component communication\label{component-map}}

The \emph{component map}
of an app contains information regarding how the components of this app
communicate with each other, and how they communicate with components from other
apps. Section \ref{component-map-generation} explains how to generate a
component map file for an app.

\subsection{Semantics of a component map}
The component map matches \sendIntent{} calls to
declarations of the \onReceive{} methods they implicitly invoke. A \sendIntent{}
call may be paired with more than one \onReceive{} declaration. Each such pair
indicates that the two components, possibly from different apps, may communicate
through intents. The set of pairs of communicating components is conservative,
that is it includes all possible pairs of methods that might communicate.

\subsection{Syntax of a component map file}
Each line of a component map file specifies one intent-sending method in the app
and all components in the app that may receive intents that it sends.

For example, the following line

\code{com.package.ActivityA.foo() -> com.package.ActivityB,
com.package.ActivityC} 

\noindent
indicates that an intent sent in the method \code{foo()} from 
\code{ActivityA} may be received by the components whose fully-qualified class 
names are \code{com.package.ActivityB} and \code{com.package.ActivityC}.

\subsection{Using a component map file}
It is recommended to name the component map file as \code{component-map} and to
put it in the top level app directory. By doing so the ant target will use it
automatically when running:

\code{ant check-intent}

\noindent
Alternatively it is possible to pass the component map file path as an option
to ant:

\code{ant -DcomponentMap=\textit{mycomponentmapfile} check-intent}

\section{Intent types\label{intent-types}}
An intent contains a map from string keys to arbitrary values. Consider an
intent variable \<i>. Data can be added to the map of \<i> by
the sender component with \<i.putExtra("key",val)> and then retrieved by the
receiver component with \<i.getExtra("key",default)>. An intent type is an
approximation to the keys that may be in the intent object at run time and to
the type of the values that those keys may map to. The type qualifiers used to
represent an intent type are \<@IntentMap> and \<@Extra>.

\subsection{Syntax}
The type qualifier \<@IntentMap> on an intent variable's type indicates the
types of the key/value mappings that are permitted to be accessed via
\<putExtra> and \<getExtra> calls.
An \<@IntentMap> type qualifier contains a set of \<@Extra> type qualifiers. An
\<@Extra> type qualifier contains a key \|k|, a source type \|source|, and a 
sink type \|sink|. This means that the key \|k| maps to a value of source type 
\|source| and sink type \|sink|. Consider the declaration below:

\begin{Verbatim}
@IntentMap({@Extra(key = "k1", source = {FILESYSTEM}, sink = {INTERNET}),
            @Extra(key = "k2", source = {FILESYSTEM}, sink = {DISPLAY})}) 
Intent i = new Intent();
\end{Verbatim}

The variable \<i> is annotated with an \<@IntentMap> type containing a set of
two \<@Extra> types, allowing this variable to be accessed via
\<i.putExtra("k1",...)>, \<i.putExtra("k2",...)>, \<i.getExtra("k1")> and
\<i.getExtra("k2")>. No other keys are allowed to be accessed via
\<putExtra> or \<getExtra> calls. 

Each intent variable's type must have only one \<@IntentMap> type qualifier. Two
different \<@Extra> type qualifiers in the same \<@IntentMap> may not have the
same key \|k|.

Every \onReceive{} method has an intent formal parameter.
Below is an example of an annotated intent formal parameter for the \onReceive{}
method \<setIntent>:

\begin{Verbatim}
@Override
public void setIntent(@IntentMap({
        @Extra(key = "location", source = {ACCESS_FINE_LOCATION }, sink = {})
                }) Intent newIntent) {
    super.setIntent(newIntent);
}
\end{Verbatim}

When instantiating an intent object in method bodies, the instantiation
expression must be annotated using a cast with the qualifiers \<@IntentMap> and
\<@Extra>. Below is an example:

\begin{Verbatim}
void foo() {
    Intent i = (@IntentMap({
            @Extra(key = "k1", source = {FILESYSTEM}, sink = {INTERNET}),
            @Extra(key = "k2", source = {FILESYSTEM}, sink = {DISPLAY})}) Intent)
                new Intent();
}
\end{Verbatim}


\subsection{Semantics}
If variable \<i> is declared to have an intent type \textit{T}, then two facts must be
true. (1) For any key \<k> that is accessed at run time in \<i>, it must be
listed in \textit{T}. That is, the keys of \<i> that are accessed are
a subset of \textit{T}'s keys. It is
permitted for the run-time value of variable \<i> to have fewer keys than those
listed by its type. It is also permitted for the run-time value of variable
\<i> to have more keys than those listed by its type but they may not be
accessed. (2) For any key \|k| that is a key in the run-time value of \<i>,
the value mapped by \|k| in the value has type mapped by \|k| in \textit{T}. 
This can be more concisely expressed as $\forall k.i[k] <: T[k]$. As permitted
by object-oriented typing, the run-time class of \<i[\|k|]> may be a subtype of \textit{T}.

As an example, consider the declarations and method calls below:

\begin{Verbatim}
@IntentMap({@Extra(key = "picture", source = {FILESYSTEM}, sink = {INTERNET}),
            @Extra(key = "location", source = {FILESYSTEM}, sink = {DISPLAY})}) 
Intent i = new Intent();

@Source(FILESYSTEM) @Sink(INTERNET) File getPicture() {...}
@Source(ACCESS_FINE_LOCATION) @Sink(INTERNET) String getLocation() {...}

void fillIntent() {
    i.putExtra("picture", getPicture());        // Legal.
    i.putExtra("someRandomKey", getPicture());  // Violates requirement (1).
    i.putExtra("location", getLocation());      // Violates requirement (2).
    ...
}

void processDataFromIntent() {
    // pic will have source type FILESYSTEM and sink type INTERNET.
    File pic = i.getExtra("picture", null);               // Legal.
    // loc will have source type FILESYSTEM and sink type DISPLAY.
    String loc = i.getStringExtra("location", null);      // Legal.
    Object randomObject = i.getExtra("someRandomKey");    // Violates requirement (1)
    ...
}
\end{Verbatim}

\noindent
The type of variable \<i> indicates that this object may contain up to two
elements in its map which can be accessed, one with key \<"picture">,
source type \<FILESYSTEM>, and sink type
\<INTERNET>, and another with key \<"location">, source type \<FILESYSTEM>, and sink
type \<DISPLAY>. This object may contain more elements but they cannot be
accessed. The method calls in the method \<fillIntent> shows that it is only
valid to invoke putExtra if the value passed as argument is a subtype of the
declared type for the corresponding key. In the method \<processDataFromIntent>,
the variables \<pic> and \<loc> will have their source and sink types
inferred from the type of \<i>. Trying to access key \<"someRandomKey"> violates
requirement (1).


\subsubsection{Subtyping}
Intent type \|T1| is a subtype of intent type \|T2| if the key set of 
\|T2| is a subset of the key set of \|T1| and, for each key \|k| in both 
\|T1| and \|T2|, \|k| is mapped to the exact same type, that is, 
\|T1[k]| = \|T2[k]|.

\subsubsection{sendIntent calls and copyable rule}
A \sendIntent{} call can be viewed as an invocation of one or more \onReceive{} methods.
A \sendIntent{} call type-checks if its intent argument is
copyable to the formal parameter of each corresponding \onReceive{} methods.
Copyable is a subtyping-like relationship with the weaker requirement: 
\<T1[k]> <: \<T2[k]> instead of \<T1[k]> = \<T2[k]>.
This is sound because the Android system passes a copy of the intent argument to
\onReceive{}, so aliasing is not a concern.

\subsubsection{Overriding and calling onReceive methods\label{override-onreceive}}
Every \onReceive{} method has a parameter of type \<Intent>, and this
parameter must be annotated with \<@IntentMap> and \<@Extra>.

The normal Java overriding rules do not apply to declarations of \onReceive{}. The
type of the formal parameter of \onReceive{} is not restricted by the type of the
parameter in the overridden declaration. This is allowable because by convention
\onReceive{} is never called directly but rather is only called by the Android
system via a \sendIntent{} method call. \TheIntentChecker{} prohibits direct
calls to \onReceive{} methods.

There is a peculiarity for the \onReceive{} method in Activities, \<setIntent>.
Every Activity that calls the method \<getIntent> must override both methods
\<setIntent> and \<getIntent>. The return type of \<getIntent> must be annotated
with the same type as the formal parameter of \<setIntent>, so that when the
method \<getIntent> is called the correct type is returned.

%%  LocalWords:  \sendIntent{} \onReceive{} typeof callee app startService
%%  LocalWords:  startActivity BroadcastReceiver sendBroadcast setIntent
%%  LocalWords:  onBind onStartCommand onReceive foo ActivityA putExtra k1
%%  LocalWords:  DcomponentMap mycomponentmapfile getExtra IntentMap k2 T1
%%  LocalWords:  FILESYSTEM fillIntent processDataFromIntent pic loc T2
%%  LocalWords:  someRandomKey sendIntent getIntent
