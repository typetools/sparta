\htmlhr


\newcommand{\method}[1]{\paragraph{#1}}
\newcommand{\annomethod}[1]{\small{\texttt{#1}}\newline}
\chapter{Tutorial\label{tutorial}}

This chapter demonstrates how to annotate an existing app, ContactManger,
 where the annotator did not develop the app and the app is assumed to be benign.
   ContactManger allows the user to view and create contacts that are associate 
   with different accounts.  


\section{Set up}
Download the ContactManager app here: \url{http://types.cs.washington.edu/sparta/tutorial/ContactManager.tgz}
Install the Information Flow Checker and set up ContactManger to use the Information Flow Checker. 
See \secref{sec:antsetup}  and \secref{sec:install} for instructions. Also, install the inference framework
see \secref{} for details.

Infer types, \<ant infer>.
Then, run the Information Flow Checker, \<ant check-flow>, if the output is similar to 
the output shown below, then your setup is correct.  (You should get 14 warnings.)

\begin{Verbatim}
Buildfile: .../ContactManager/build.xml
...
check-flow-pp:
[jsr308.javac] Compiling 4 source files to .../ContactManager/bin/classes
[jsr308.javac] javac 1.8.0-jsr308-1.8.2
[jsr308.javac] warning: The following options were not recognized by any processor: '[componentMap, ignorenr, prettyPrint]'
[jsr308.javac] .../ContactManager/src/com/example/android/contactmanager/ContactAdder.java:195: warning: [argument.type.incompatible] incompatible types in argument.
[jsr308.javac]                 .withValue(ContactsContract.CommonDataKinds.StructuredName.DISPLAY_NAME, name)
[jsr308.javac]                                                                                          ^
[jsr308.javac]   found   :  { USER_INPUT  ->   }  String
[jsr308.javac]   required:  {   ->  CONTENT_PROVIDER }  Object
\end{Verbatim} 

\section{Drafting a flow policy}

The Information Flow Checker outputs a file, \<sparta-out/forbiddenFlows.txt>, that lists all the flows
 it found in the app that are not allowed by the current flow policy.  For this app, 
 \<forbiddenFlows.txt> is shown below. 

\begin{Verbatim} 
# Flows currently forbidden
 ANY -> ANY
 CONTENT_PROVIDER -> ANY
 USER_INPUT -> WRITE_LOGS, CONTENT_PROVIDER
\end{Verbatim} 
Because this app does not yet have a flow policy, this file lists all the flows that the 
Information Flow Checker was able to detect.  This is not a complete list of flows in the program,
because the Information Flow Checker issued warnings, but it offers a good starting point 
for the flow policy. 

Create a file named \<flow-policy> in the ContactManger directory.  Copy all the flows that   
do not flow to or from \perm{ANY}.  These flows should not be copied because 
they are too permissive.  So, for this app the initial flow policy is shown below.
\begin{Verbatim} 
 USER_INPUT -> WRITE_LOGS, CONTENT_PROVIDER
\end{Verbatim}

 Run the Information Flow Checker again.  (Because you named your flow policy 
\<flow-policy>, the Information Flow Checker will automatically read it.)  The Information Flow Checker should
now only report 10 warnings.  The \<forbiddenFlows.txt> file should also have changed as shown
below.

\begin{Verbatim}
# Flows currently forbidden
 ANY -> ANY
 CONTENT_PROVIDER -> ANY
\end{Verbatim}
 
 Since both of these flows contain \perm{ANY}, there is nothing to add 
 to the flow policy.  However, the flow policy may not be complete, because the Information Flow Checker
 issued 10 warnings.  



\section{Adding Annotations}

Next, annotate the code to ensure the flow policy correctly represents the flows
in the app.  One way to annotate an unfamiliar app, is to work through each warning
one by one. Correct it by adding annotations, re-running the Information Flow Checker, 
and then correct the next warning.  In general, 
Information Flow Checker warnings are written as shown below.

\begin{Verbatim}  
../SomeClass.java:line number: warning: some types are incompatible 
             source code causing warning
                     a caret (^) pointing to the location of the warning.
   found   : Qualified type found by the Information Flow Checker
   required: Qualified type that the Information Flow Checker was expecting.
\end{Verbatim}
 
In order to correct a warning and correctly capture the app behavior, answer 
the following questions for each warning.
\begin{enumerate}
\item Why are the found and required annotations those listed in the warning message?
   \begin{itemize}
    \item Is the annotation explicitly written in the source code? 
    \item Is the annotation from an API method that was annotated in stub file? \secref{sec:apispecs}
    \item Is the annotation inferred? \secref{sec:type-inference}
    \item Is the annotation defaulted? \secref{sec:unannotated-types}
    \end{itemize}
\item Why is the found type not a subtype of the required type? \secref{sec:subtyping}
  \begin{itemize}
   \item Does the found type have more or different found sources than required?
   \item Does the found type have less or different found sinks than required? 
   \end{itemize}
\item What annotation or annotations correctly capture the behavior of the app at
  this location? (In other words, what annotation will make the found type a 
  subtype of the required type?)
   \begin{itemize}
   \item Add only a source or a sink annotation
   \end{itemize}
\end{enumerate}

This tutorial only covers incompatibility warnings; see section \secref{errors} for 
other warnings and how to handle them.

\subsection{Warning 1}
Run the Information Flow Checker again.

\begin{Verbatim}  
.../ContactManager/src/com/example/android/contactmanager/ContactAdder.java:309:
 warning: incompatible types in assignment.
             mName = name;
                     ^
   found   : @Sink(CONDITIONAL) @Source(ANY) String
   required: @Sink({CONDITIONAL, DISPLAY, WRITE_LOGS, CONTENT_PROVIDER})  @Source(LITERAL) String
\end{Verbatim}
This is an ``incompatible types in assignment'' error.  It means that the type
of the left hand side, the found type, is not a subtype of the right hand side,
the required type.
 
\begin{enumerate}

\item \textbf{Where do the found  and required types come from?}

The type of \<name> is they found type and the type of \<mName> is
shown as \emph{required}.  

 \begin{itemize}
    \item Was it explicitly annotated in the source code? No, we have not added 
    any annotations.
    \item Is it from an API method that was annotated in stub file? No, \<mName> 
    is a field and \<name> is a parameter to this method.
    \item Was it inferred from an assignment? No, neither variable was assigned 
    perviously in this method. 
    \item Was it defaulted? Yes.
    \end{itemize}

 According to the defaulting rules explained in \secref{sec:unannotated-types} fields, like \emph{mName} 
 are annotated with \<@Source(LITERAL)> and the sink annotation is defaulted base on the  
 flow policy.  For this app, literals are allowed to flow to \perm{DISPLAY}, 
 \perm{WRITE\_LOGS}, and\perm{CONTENT\_PROVIDER}, so the sink annotation is 
 defaulted to \<@Sink(\{DISPLAY, WRITE\_LOGS, and CONTENT\_PROVIDER\})>.
 
 Method parameters, like \emph{name}, are annotated with \<@Sink(CONDITIONAL)> by 
 default and the source is defaulted base on what is allowed to flow to conditionals.
   \flow{ANY}{CONDTIONAL} is added to the flow policy by default, so the default types
    for fields in this class is \<@Sink(CONDITIONAL)> \<@Source(ANY)>.

 
 \item\textbf{ Why is the found type not a subtype of the required type?}
The found type has more sources than the required type. (Remember that \perm{ANY}
is the set of all Flow Permissions). Also, the \emph{found} type has fewer sinks 
than the required type.

\item\textbf{What annotation or annotations would make the found type a subtype of the required?}
One of the defaults described above needs to be overridden by annotating the 
 either to the type of \<name> or the type of \<mName>. So which type should be 
 annotated? Because the current type of \<name> allows more flows than the type
 of \<mName>, its annotation should be changed so that it is a sub type of \<mName>.
  The type of \<name> should therefore be changed to \<@Source(LITERAL)>. The 
  sink annotation will be defaulted base on the flow policy and will match the 
  sink annotation from the required type.

 \annomethod{public AccountData(\textbf{@Source(LITERAL)} String name, AuthenticatorDescription description) }

 This warning could also have been corrected by annotating \<mName> with \<@Sink(CONDITIONAL)>.
   In general, you should pick that least permissive one.  Literals are only 
   allowed to flow to four sinks, but any source can flow to conditionals, so
 \<@Source(LITERAL)> is less permissive than \<@Sink(CONDITIONAL)>.
 \end{enumerate}
 Run the Information Flow Checker again.  Only four errors should be issued.
 
 \subsection{Warning 2}
 \begin{Verbatim}
.../ContactManager/src/com/example/android/contactmanager/ContactAdder.java:387: 
warning: incompatible types in return.
             return convertView;
                    ^
   found   : @Sink(CONDITIONAL) @Source(ANY) View
   required: @Sink({CONDITIONAL, DISPLAY, WRITE_LOGS, CONTENT_PROVIDER}) @Source(LITERAL) View
\end{Verbatim} 
This is an ``incompatible types in return'' error.  It means that the type
of returned by this method, the found type, is not a subtype of the declared 
return type of this method, the required type.
 \begin{enumerate}

\item\textbf{Where do the found and required types come from?}
\<convertView> is a parameter that is defaulted to \<@Sink(CONDITIONAL)> and 
the required type is the return type of this method. Because there is not an
annotation on the return type of this method, its type is defaulted to \<@Source(LITERAL)>.

  \item\textbf{ Why is the found type not a subtype of the required type?}
The found type has more sources than the required type.
 Also, the \emph{found} type has fewer sinks 
than the required type.

  \item\textbf{What annotation or annotations would make the found type a
   subtype of the required?} This warning is similar to the first warning and 
   should also be fixed by annotating the found type with \<@Source(LITERAL)>.\newline
  
        \annomethod{public View getDropDownView(int position, \textbf{@Source(LITERAL)} View convertView, ViewGroup parent)}
   \end{enumerate}

   Run the Information Flow Checker again.  Only three errors should be issued.


  \subsection{Warning 3 }
   \begin{Verbatim}
.../ContactManager/src/com/example/android/contactmanager/ContactManager.java:74: 
warning: incompatible types in argument.
                 Log.d(TAG, "mShowInvisibleControl changed: " + isChecked);
                                                              ^
   found   : @Sink(CONDITIONAL) @Source(ANY) String
   required: @Sink(WRITE_LOGS) @Source({LITERAL, USER_INPUT, CONTENT_PROVIDER}) String
\end{Verbatim} 
This is an ``incompatible types in argument'' error.  It means that the type
of argument, the found type, is not a subtype of the formal 
parameter type of the method called, the required type.
 \begin{enumerate}

\item\textbf{Where do the found and required types come from?}
The found type is the type of the concatenation of the literal string and \<isChecked>.  The
resulting type of a concatenation is the least upper bounds, LUB, of the two types.   The LUB
of a flow type is simply 
 the union of sources and the intersection of sinks.  The string literal is of type 
\<@Source(LITERAL)>.  \<isChecked> is a parameter of a methods that overrides an
Android API method.  The types of overriding methods must match those of the overridden
method, so check the stub files for this method.  \<isChecked> is annotated with 
\<@Source(USER\_INPUT)>  in the stub file, so it must be annotated the same here.\newline

       \annomethod{public void onCheckedChanged(CompoundButton buttonView, \textbf{@Source(USER\_INPUT)} boolean isChecked)}
 \end{enumerate}

Run the Information Flow Checker; there should be 2 warnings.
  
  \subsection{Warning 4 }
   \begin{Verbatim}
.../ContactManager/src/com/example/android/contactmanager/ContactManager.java:77: 
warning: incompatible types in assignment.
                 mShowInvisible = isChecked;
                                  ^
   found   : @Sink({CONDITIONAL, WRITE_LOGS, CONTENT_PROVIDER}) @Source(USER_INPUT) boolean
   required: @Sink({CONDITIONAL, DISPLAY, WRITE_LOGS, CONTENT_PROVIDER}) @Source(LITERAL) boolean
    \end{Verbatim} 

 \begin{enumerate}

\item\textbf{Where do the found and required types come from?}
   The found type is a parameter that is annotated and the required type is a field that was 
   defaulted.
\item  \textbf{ Why is the found type not a subtype of the required type?}
  The found source is \perm{USER\_INPUT} and the required source is \perm{LITERAL}. 
\item  \textbf{What annotation or annotations would make the found type a subtype of the required?}
  The required type has already been annotated, so it should not be changed; therefore, the
  found type should be annotated with \<@Source(USER\_INPUT)>.\newline
  
  \annomethod{private \textbf{@Source(USER\_INPUT)} boolean mShowInvisible;}
  
   \end{enumerate}

Run the Information Flow Checker; there should be 3 warnings.  Two of the warnings are because of the
annotation that was just added.

  \subsection{Warning 5}
   \begin{Verbatim}
.../ContactManager/src/com/example/android/contactmanager/ContactManager.java:63: 
warning: incompatible types in assignment.
         mShowInvisible = false;
                          ^
   found   : @Sink({CONDITIONAL,DISPLAY, WRITE_LOGS, CONTENT_PROVIDER}) @Source(LITERAL) boolean
   required: @Sink({CONDITIONAL, WRITE_LOGS, CONTENT_PROVIDER}) @Source(USER_INPUT) boolean
    \end{Verbatim} 

 \begin{enumerate}

\item\textbf{Where do the found and required types come from?}
   The found type is from the boolean literal.  The required type is from an annotated field.
  \item\textbf{ Why is the found type not a subtype of the required type?}
  The found source is \perm{LITERAL} and the required type is \perm{USER\_INPUT}
  \item\textbf{What annotation or annotations would make the found type a subtype of the required?}
  The annotation on the boolean literal cannot be changed, so the annotation of the field must be changed.  The annotation is \<@Source(USER\_INPUT)>, so we can add the 
  \perm{LITERAL} to the list of sources.\newline
  
    \annomethod{    private \textbf{@Source(\{USER\_INPUT, LITERAL\})} boolean mShowInvisible;}
      \end{enumerate}

    Run the Information Flow Checker; there should be 1 warning.  The last change fixed two warnings.
    
  \subsection{Warning 6}
   \begin{Verbatim}
.../ContactManager/src/com/example/android/contactmanager/ContactManager.java:118: 
warning: incompatible types in return.
         return managedQuery(uri, projection, selection, selectionArgs, sortOrder);
                            ^
   found   : @Sink({CONDITIONAL, DISPLAY, WRITE_LOGS, CONTENT_PROVIDER}) @Source(CONTENT_PROVIDER) Cursor
   required: @Sink({CONDITIONAL, DISPLAY, WRITE_LOGS, CONTENT_PROVIDER}) @Source(LITERAL) Cursor
    \end{Verbatim} 

 \begin{enumerate}

\item\textbf{Where do the found and required types come from?}
   The found type is from the return type of an API method that was annotated in a stub file.  The required type is the return type of this method, \<getContacts()>, which is defaulted. 
  \item\textbf{ Why is the found type not a subtype of the required type?}
  The found source is \<@Source(CONTENT\_PROVIDER)> and the required source is 
  \<@Source(LITERAL>.
  \item\textbf{What annotation or annotations would make the found type a subtype of the required?}
  Because the found type is from a stub file annotation, it cannot be changed. So, the return type of this method, \<getContacts()> must be annotated with 
  \<@Source(CONTENT\_PROVIDER>.\newline
  
             \annomethod{ private \textbf{@Source(CONTENT\_PROVIDER)}
             Cursor getContacts()}
              \end{enumerate}

    Run the Information Flow Checker; there should be no warnings.
  \section{Correctly annotated app}  
     Now that the Information Flow Checker no longer reports any warnings, it guarantees that
     ContactManger only contains the information flows in the flow policy.  

 
%
%
%
%
